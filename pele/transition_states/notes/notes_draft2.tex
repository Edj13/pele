\documentclass[a4paper]{article}

%\usepackage[numbers]{natbib}
\usepackage[top=2.5cm,bottom=2.5cm]{geometry}
\usepackage{graphicx}
\usepackage{hyperref}
%\usepackage{amsmath}
%\bibliographystyle{naturemagurl}
%\bibliographystyle{plainnat}


\title{Notes on finding the lowest eigenvector}
\author{Jacob Stevenson}
\date{\today}

\begin{document}
\maketitle

This describes a method for computing the smallest eigenvector of a Hessian matrix
$H_{ij} = \partial E(\vec{x}) / (\partial{x_i}\partial x_j)$ 
of an energy function $E(\vec{x})$.  In this method the gradients $\vec{g} = \partial
E(\vec{x}) / \partial \vec{x}$
of the energy function are used, but the exact Hessian is not.  The lowest eigenvector is found
by estimating the curvature at a give point $\vec{x}$ along a direction $\vec{v}$ using finite differences and then
using an optimization algorithm to minimizing the curvature with respect
to $\vec{v}$.  The curvature
is estimated by the the central differences formula, which is accurate up to
order $\delta^2$ where $\delta$ is a small parameter.
\begin{equation}
\mu(\vec{x}, \vec{v}) = \frac{1}{\delta} 
\left[ \vec{g}(\vec{x} + \frac{\delta}{2}  \vec{v}) - 
\vec{g}(\vec{x} - \frac{\delta}{2} \vec{v}) \right] \cdot \vec{v}
\end{equation}
If N is the dimension of the original space then this is an optimization
problem in N-1 dimensions because of the constraint that $\left|\vec{v} \right|
= 1$.

The gradients of the above function can be computed analytically to aid in the
optimization.
\begin{equation}
\frac{\partial \mu(\vec{x}, \vec{v})} {\partial v_k} = 
\frac{1}{\delta} 
\sum_i \frac{\partial v_i}{\partial v_k}
\left[ g_i(\vec{x} + \frac{\delta}{2} \vec{v}) 
- g_i(\vec{x} - \frac{\delta}{2} \vec{v}) \right]
+
\frac{1}{ \delta} 
\sum_i v_i
\frac{\partial}{\partial v_k}
\left[ g_i(\vec{x} + \frac{\delta}{2} \vec{v}) 
- g_i(\vec{x} - \frac{\delta}{2} \vec{v}) \right]
\end{equation}
In the next we use
\begin{equation}
\frac{\partial g_i(\vec{x} + \frac{\delta}{2} \vec{v})}{ \partial v_k} 
= 
\frac{\delta}{2} \sum_j
\frac{\partial g_i(\vec{x} + \frac{\delta}{2} \vec{v})}{ \partial x_j}
\frac{\partial  v_j} { \partial v_k}
= \frac{\delta}{2} 
\frac{\partial g_i(\vec{x} + \frac{\delta}{2} \vec{v})}{ \partial x_k} 
\end{equation}
where $\partial v_i / \partial v_k$ is the kroniker delta $\delta_{ik}$.
\begin{equation}
\frac{\partial \mu(\vec{x}, \vec{v})} {\partial v_k} = 
\frac{1}{\delta} 
\left[ g_k(\vec{x} + \frac{\delta}{2} \vec{v}) 
- g_k(\vec{x} - \frac{\delta}{2} \vec{v}) \right]
+
\frac{1}{2} 
\sum_i v_i
\frac{\partial}{\partial x_k}
\left[ g_i(\vec{x} + \frac{\delta}{2} \vec{v}) 
+ g_i(\vec{x} - \frac{\delta}{2} \vec{v}) \right]
\end{equation}
To second order in $\delta$ we can replace the second term with
\begin{equation}
\frac{\partial \mu(\vec{x}, \vec{v})} {\partial v_k} = 
\frac{1}{\delta} 
\left[ g_k(\vec{x} + \frac{\delta}{2} \vec{v}) 
- g_k(\vec{x} - \frac{\delta}{2} \vec{v}) \right]
+
\sum_i v_i
\frac{\partial g_i(\vec{x})}{\partial x_k}
\end{equation}
We can now use the eigenvalue equation $\vec{v} \cdot H = \mu \vec{v}$ to further simplify the second term
\begin{equation}
\frac{\partial \mu(\vec{x}, \vec{v})} {\partial v_k} = 
\frac{1}{\delta} 
\left[ g_k(\vec{x} + \frac{\delta}{2} \vec{v}) 
- g_k(\vec{x} - \frac{\delta}{2} \vec{v}) \right]
+
\mu v_k
\end{equation}
or
\begin{equation}
\frac{\partial \mu(\vec{x}, \vec{v})} {\partial \vec{v}} = 
\frac{1}{\delta} 
\left[ \vec{g}(\vec{x} + \frac{\delta}{2} \vec{v}) 
- \vec{g}(\vec{x} - \frac{\delta}{2} \vec{v}) \right]
+
\mu \vec{v}
\end{equation}

We want to maintain the constraint $\left|\vec{v} \right| = 1$, so we subtract out the parallel component of the derivative
\begin{equation}
\frac{\partial \mu(\vec{x}, \vec{v})} {\partial \vec{v}} - 
\left(
\frac{\partial \mu(\vec{x}, \vec{v})} {\partial \vec{v}} \cdot \vec{v}
\right) \vec{v}
= 
\frac{1}{\delta} 
\left[ \vec{g}(\vec{x} + \frac{\delta}{2} \vec{v}) 
- \vec{g}(\vec{x} - \frac{\delta}{2} \vec{v}) \right]
+
\mu \vec{v}
-
\left(
\frac{1}{\delta} 
\left[ \vec{g}(\vec{x} + \frac{\delta}{2} \vec{v}) 
-\vec{g}(\vec{x} - \frac{\delta}{2} \vec{v}) \right] \cdot \vec{v}
+
\mu \right)
\vec{v}
\end{equation}
The first term in the parenthesis on the right is simply $\mu$ which gives us
finally
\begin{equation}
\frac{\partial \mu(\vec{x}, \vec{v})} {\partial \vec{v}} - 
\left(
\frac{\partial \mu(\vec{x}, \vec{v})} {\partial \vec{v}} \cdot \vec{v}
\right) \vec{v}
= 
\frac{1}{\delta} 
\left[ \vec{g}(\vec{x} + \frac{\delta}{2} \vec{v}) 
- \vec{g}(\vec{x} - \frac{\delta}{2} \vec{v}) \right]
-
\mu \vec{v}
\end{equation}
This equation is off by a factor of two. The correct answer is twice the above.  David uses the correct equation and I've checked it with numerical derivatives. Possible sources of the missing factor of 2 in order of most likely
\begin{enumerate}
\item If you think of it as rotating a dimer there are two ends of the dimer ($(\vec{x} + \frac{\delta}{2} \vec{v})$ and $(\vec{x} - \frac{\delta}{2} \vec{v})$) and each contributes the above force.  This seems like the most likely source, but I just can't wrap my head around it.  Why does simply differentiating $\mu$ with respect to $\vec{v}$ not work?

\item  Because of the constraint $\left|\vec{v} \right| = 1$ I need to do the
derivative in curvilinear coordinates.  This should be thought of as a rotation, and simple derivatives is not sufficient.  

\item I throw away something important in going from equation 4 to 5.

\item Note that the second term in the above equations ends up completely parallel to $\vec{v}$ and therefore is disposed of when subtracting out the parallel component.  The mistake (I guess) must be in the first term

\end{enumerate}


\end{document}
